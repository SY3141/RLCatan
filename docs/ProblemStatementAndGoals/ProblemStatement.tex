\documentclass{article}


\usepackage{tabularx}
\usepackage{booktabs}

\title{Problem Statement and Goals\\\progname}

\author{\authname}

\date{}

%% Comments

\usepackage{color}

\newif\ifcomments\commentstrue %displays comments
%\newif\ifcomments\commentsfalse %so that comments do not display

\ifcomments
\newcommand{\authornote}[3]{\textcolor{#1}{[#3 ---#2]}}
\newcommand{\todo}[1]{\textcolor{red}{[TODO: #1]}}
\else
\newcommand{\authornote}[3]{}
\newcommand{\todo}[1]{}
\fi

\newcommand{\wss}[1]{\authornote{magenta}{SS}{#1}} 
\newcommand{\plt}[1]{\authornote{cyan}{TPLT}{#1}} %For explanation of the template
\newcommand{\an}[1]{\authornote{cyan}{Author}{#1}}

%% Common Parts

\newcommand{\progname}{Software Engineering} % PUT YOUR PROGRAM NAME HERE
\newcommand{\authname}{Team 8, RLCatan
\\ Rebecca Di Filippo
\\ Jake Read
\\ Matthew Cheung
\\ Sunny Yao} % AUTHOR NAMES

\usepackage{hyperref}
    \hypersetup{colorlinks=true, linkcolor=blue, citecolor=blue, filecolor=blue,
                urlcolor=blue, unicode=false}
    \urlstyle{same}
                                


\begin{document}

\maketitle

\begin{table}[hp]
\caption{Revision History} \label{TblRevisionHistory}
\begin{tabularx}{\textwidth}{llX}
\toprule
\textbf{Date} & \textbf{Developer(s)} & \textbf{Change}\\
\midrule
Sept. 17th & Matthew & Added preliminary problem statement for sections 1-1.4\\
Sept. 21st & Matthew & Updated problem statement sections 1-1.4\\
... & ... & ...\\
\bottomrule
\end{tabularx}
\end{table}

\section{Problem Statement}\label{sec:problem-statement}

\subsection{Problem}
The project aims to create an AI-enabled decision-support tool for the board game \emph{Settlers of Catan}. The core problem is that \emph{Catan's} complexity, with its numerous possible moves and game states, makes it challenging for human players to calculate optimal strategies. This challenge arises from the vast number of game states, the partially observable information (e.g., opponents' hidden cards), and the stochastic nature of dice rolls. This project seeks to overcome these limitations by using an AI reinforcement learning (RL) model to provide strategic guidance. The final product is a digital tool that uses computer vision to observe a physical game and offers strategic suggestions to the player via a device.

\subsection{Inputs and Outputs}
The system's high-level inputs are the game state information of a \emph{Catan} match, including all board and player data. This information is captured in real-time by a camera and processed to create a digital representation of the board. The high-level outputs are strategic recommendations for the player, delivered directly to their device. The AI will suggest the best moves and can also simulate gameplay against various AI bots. The tool also provides post-game advice, identifying past decisions that could have been made differently to alter the game's outcome.

\subsection{Stakeholders}\label{subsec:stakeholders}
The primary stakeholders for this project are:
\begin{itemize}
    \item Players of \emph{Settlers of Catan}: The end-users who will use the AI for in-game and post-game analysis to improve their skills, or train against various skilled AI bots.
    \item Dr. Istvan David: The project supervisor who provides guidance, expertise, and oversight.
    \item The Project Team: The developers responsible for designing, implementing, and testing the software.
    \item The Department of Computing and Software (CAS) at McMaster University: The organization hosting the project, which benefits from the academic and technical achievements of its students and faculty.
\end{itemize}

\subsection{Environment}
The project requires a hardware and software environment to support its modules. On the software side, the system will need a game simulator for training, a reinforcement learning framework for the AI, a computer vision library to process video, and a user-facing application for the player's device. On the hardware side, the project requires a computer with a processor and GPU to run the AI, devices for the players to use the application and stream video of the game board, and a physical \emph{Catan} board game for testing.

\section{Goals}\label{sec:goals}
The following are the major goals for the project:
\begin{itemize}
    \item Create a simulation environment for \emph{Catan}, encoding game rules such that in a given state, a set of potential moves is returned.
    \item Devise starting states/confer with experts to find some initial board setups and corresponding strategies to aid the model’s early learning.
    \item Utilize the simulation environment to train a deep reinforcement learning model to play \emph{Catan} to at least the level of the average human player.
    \item Use computer vision/sensors to transfer the current state of a physical table-top \emph{Catan} game to our simulation (Create a digital twin).
    \item Send the move generated by the model back to the user in some form.
    \item Develop a visualisation of the game on the digital twin’s side to help explain actions.
\end{itemize}

\section{Stretch Goals}\label{sec:stretch-goals}
In addition to the project’s primary goals the following are stretch goals we will aim to complete if possible:
\begin{itemize}
    \item Construct personas relating to various kinds of players and the strategies they will typically utilize in the game’s trading system in order to improve the model’s predictions.
    \item Improve the model to the extent that it performs better than the average \emph{Catan} player, perhaps even finding strategies unknown to current competitive players.
    \item Generate explanations of “what could’ve been” following a game using an LLM\@.
    \item Further augment the physical board to simulation transfer via the use of multiple camera angles or smart glasses.
\end{itemize}

\section{Extras}\label{sec:extras}

\wss{For CAS 741: State whether the project is a research project. This
designation, with the approval (or request) of the instructor, can be modified
over the course of the term.}

\wss{For SE Capstone: List your extras.  Potential extras include usability
testing, code walkthroughs, user documentation, formal proof, GenderMag
personas, Design Thinking, etc.  (The full list is on the course outline and in
Lecture 02.) Normally the number of extras will be two.  Approval of the extras
will be part of the discussion with the instructor for approving the project.
The extras, with the approval (or request) of the instructor, can be modified
over the course of the term.}

\newpage{}

\section*{Appendix --- Reflection}

\wss{Not required for CAS 741}

The purpose of reflection questions is to give you a chance to assess your own
learning and that of your group as a whole, and to find ways to improve in the
future. Reflection is an important part of the learning process.  Reflection is
also an essential component of a successful software development process.  

Reflections are most interesting and useful when they're honest, even if the
stories they tell are imperfect. You will be marked based on your depth of
thought and analysis, and not based on the content of the reflections
themselves. Thus, for full marks we encourage you to answer openly and honestly
and to avoid simply writing ``what you think the evaluator wants to hear.''

Please answer the following questions.  Some questions can be answered on the
team level, but where appropriate, each team member should write their own
response:


\begin{enumerate}
    \item What went well while writing this deliverable? 
    \item What pain points did you experience during this deliverable, and how
    did you resolve them?
    \item How did you and your team adjust the scope of your goals to ensure
    they are suitable for a Capstone project (not overly ambitious but also of
    appropriate complexity for a senior design project)?
\end{enumerate}  

\end{document}
