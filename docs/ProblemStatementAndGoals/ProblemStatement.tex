\documentclass{article}


\usepackage{tabularx}
\usepackage{booktabs}

\title{Problem Statement and Goals\\\progname}

\author{\authname}

\date{}

\input{../Comments}
%% Common Parts

\newcommand{\progname}{Software Engineering} % PUT YOUR PROGRAM NAME HERE
\newcommand{\authname}{Team 8, RLCatan
\\ Rebecca Di Filippo
\\ Jake Read
\\ Matthew Cheung
\\ Sunny Yao} % AUTHOR NAMES

\usepackage{hyperref}
    \hypersetup{colorlinks=true, linkcolor=blue, citecolor=blue, filecolor=blue,
                urlcolor=blue, unicode=false}
    \urlstyle{same}
                                


\begin{document}

\maketitle

\begin{table}[hp]
\caption{Revision History} \label{TblRevisionHistory}
\begin{tabularx}{\textwidth}{llX}
\toprule
\textbf{Date} & \textbf{Developer(s)} & \textbf{Change}\\
\midrule
Sept. 17th & Matthew & Added preliminary problem statement for sections 1-1.4\\
Date2 & Name(s) & Description of changes\\
... & ... & ...\\
\bottomrule
\end{tabularx}
\end{table}

\section{Problem Statement}

\subsection{Problem}
The project aims to create an AI-enabled decision-support tool for the board game Settlers of Catan. The core problem is that Catan's complexity, with its numerous possible moves and game states, makes it challenging for human players to calculate optimal strategies. This challenge arises from the vast number of game states, the partially observable information (e.g., opponents' hidden cards), and the stochastic nature of dice rolls. This project seeks to overcome these limitations by using an AI reinforcement learning (RL) model to provide strategic guidance. The final product is a digital tool that uses computer vision to observe a physical game and offers strategic suggestions to the player via a device.

\subsection{Inputs and Outputs}
The system's high-level inputs are the game state information of a Catan match, including all board and player data. This information is captured in real-time by a camera and processed to create a digital representation of the board. The high-level outputs are strategic recommendations for the player, delivered directly to their device. The AI will suggest the best moves and can also simulate gameplay against various AI bots. The tool also provides post-game advice, identifying past decisions that could have been made differently to alter the game's outcome.

\subsection{Stakeholders}
The primary stakeholders for this project are:
\begin{itemize}
    \item Players of Settlers of Catan: The end-users who will use the AI for in-game and post-game analysis to improve their skills, or train against various skilled AI bots.
    \item Dr. Istvan David: The project supervisor who provides guidance, expertise, and oversight.
    \item The Project Team: The developers responsible for designing, implementing, and testing the software.
    \item The Department of Computing and Software (CAS) at McMaster University: The organization hosting the project, which benefits from the academic and technical achievements of its students and faculty.
\end{itemize}

\subsection{Environment}
The project requires a hardware and software environment to support its modules. On the software side, the system will need a game simulator for training, a reinforcement learning framework for the AI, a computer vision library to process video, and a user-facing application for the player's device. On the hardware side, the project requires a computer with a processor and GPU to run the AI, devices for the players to use the application and stream video of the game board, and a physical Catan board game for testing.

\section{Goals}

\section{Stretch Goals}

\section{Extras}

\wss{For CAS 741: State whether the project is a research project. This
designation, with the approval (or request) of the instructor, can be modified
over the course of the term.}

\wss{For SE Capstone: List your extras.  Potential extras include usability
testing, code walkthroughs, user documentation, formal proof, GenderMag
personas, Design Thinking, etc.  (The full list is on the course outline and in
Lecture 02.) Normally the number of extras will be two.  Approval of the extras
will be part of the discussion with the instructor for approving the project.
The extras, with the approval (or request) of the instructor, can be modified
over the course of the term.}

\newpage{}

\section*{Appendix --- Reflection}

\wss{Not required for CAS 741}

The purpose of reflection questions is to give you a chance to assess your own
learning and that of your group as a whole, and to find ways to improve in the
future. Reflection is an important part of the learning process.  Reflection is
also an essential component of a successful software development process.  

Reflections are most interesting and useful when they're honest, even if the
stories they tell are imperfect. You will be marked based on your depth of
thought and analysis, and not based on the content of the reflections
themselves. Thus, for full marks we encourage you to answer openly and honestly
and to avoid simply writing ``what you think the evaluator wants to hear.''

Please answer the following questions.  Some questions can be answered on the
team level, but where appropriate, each team member should write their own
response:

\subsection*{Jake Read:}\label{subsec:jake-read-reflection}
\begin{enumerate}
    \item
        Most aspects of this deliverable went quite smoothly.
        We were able to fairly split the work, in such a way that we all contributed important sections, but were able to work in our own time, reducing time spent in meetings.
        I believe we were able to balance things in such a way that despite working on different sections, each of us still has a general understanding of the content in each part.
        What I'm happiest about is that we all seem to be working well together.
        There hasn't been any conflict between the four of us, and we were upfront about our schedules and availability and have stuck to the expectations we set.
        Whenever I ran into issues or had questions, it was easy to get a hold of someone in our Discord to help.
        The Discord server I set up has been perfect for organization, we have a general chat and various channels for sources, notes, resources, etc.
        Whenever we had any questions none of us could answer, we reached out to our TA or our supervising professor, who were happy to help.

    \item
        We had a couple pain points during the project selection.
        We knew we wanted to work with AI/ML in the project, but not what project we wanted to do, so we began by running polls in our Discord server on the various potential projects.
        After voting and a subsequent meeting, we narrowed our options down to two.
        I was mostly interested in the \emph{Catan} project, while the rest of the team was less certain which they preferred.
        We decided to schedule a meeting with the supervising profs of both projects, to get a more in-depth idea of what each one involved.
        This worked, and we ended up going with the \emph{Catan} project.
        During this whole process, one teammate missed both meetings with little explanation, and was very slow to answer messages.
        After a discussion with the group, we agreed that we were concerned by the lack of communication, and decided to gracefully part ways with our fifth member.

    \item
        Decisions surrounding scope were quite complex, as none of us had extensive prior experience with reinforcement learning.
        This made it hard to judge how long certain aspects of the project would take, so we turned to our supervisor, Professor Istvan David.
        He was able to give us rough estimates regarding the scope/viability of various goals, which was a great help.
        The nature of our project made it quite simple to separate goals however, as the design process is rather modular (build simulation, train model, return data, etc.).
        The existing project description provided in the potential projects document also helped in this regard, as certain milestones were already marked as optional, making them clear contenders for stretch goals.

\end{enumerate}


\begin{enumerate}
    \item What went well while writing this deliverable? 
    \item What pain points did you experience during this deliverable, and how
    did you resolve them?
    \item How did you and your team adjust the scope of your goals to ensure
    they are suitable for a Capstone project (not overly ambitious but also of
    appropriate complexity for a senior design project)?
\end{enumerate}  

\end{document}
