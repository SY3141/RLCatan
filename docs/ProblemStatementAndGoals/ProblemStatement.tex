\documentclass{article}


\usepackage{tabularx}
\usepackage{booktabs}

\title{Problem Statement and Goals\\\progname}

\author{\authname}

\date{}

%% Comments

\usepackage{color}

\newif\ifcomments\commentstrue %displays comments
%\newif\ifcomments\commentsfalse %so that comments do not display

\ifcomments
\newcommand{\authornote}[3]{\textcolor{#1}{[#3 ---#2]}}
\newcommand{\todo}[1]{\textcolor{red}{[TODO: #1]}}
\else
\newcommand{\authornote}[3]{}
\newcommand{\todo}[1]{}
\fi

\newcommand{\wss}[1]{\authornote{magenta}{SS}{#1}} 
\newcommand{\plt}[1]{\authornote{cyan}{TPLT}{#1}} %For explanation of the template
\newcommand{\an}[1]{\authornote{cyan}{Author}{#1}}

%% Common Parts

\newcommand{\progname}{Software Engineering} % PUT YOUR PROGRAM NAME HERE
\newcommand{\authname}{Team 8, RLCatan
\\ Rebecca Di Filippo
\\ Jake Read
\\ Matthew Cheung
\\ Sunny Yao} % AUTHOR NAMES

\usepackage{hyperref}
    \hypersetup{colorlinks=true, linkcolor=blue, citecolor=blue, filecolor=blue,
                urlcolor=blue, unicode=false}
    \urlstyle{same}
                                


\begin{document}

\maketitle

\begin{table}[hp]
\caption{Revision History} \label{TblRevisionHistory}
\begin{tabularx}{\textwidth}{llX}
\toprule
\textbf{Date} & \textbf{Developer(s)} & \textbf{Change}\\
\midrule
Sept. 17th & Matthew & Added preliminary problem statement for sections 1-1.4\\
Date2 & Name(s) & Description of changes\\
... & ... & ...\\
\bottomrule
\end{tabularx}
\end{table}

\section{Problem Statement}

\subsection{Problem}
The project aims to create an AI-enabled digital twin for the board game Settlers of Catan. The core problem is that Catan's complexity, with its numerous possible moves and game states, makes it challenging for human players to calculate optimal strategies. This project seeks to leverage AI, specifically reinforcement learning (RL), to overcome this limitation. **The AI will provide real-time decision support offering in-game advice to help players learn and improve, as well as post-game analysis to review critical moments.

\subsection{Inputs and Outputs}
The system's high-level inputs are the game state information of a Catan match. This includes the positions of roads, settlements, and cities, the resources and development cards held by each player, and the location of the robber. This information can be fed into the system in various ways, such as a manual entry of a static snapshot or **through real-time video feeds that are processed to detect the game state. The high-level outputs recommended actions for the player and the ability to simulate gameplay against one or more AI bots, each having their own unique playstyle **(e.g., longest road, largest army). **This includes suggesting the best move to make during a turn, such as where to build a road or what to trade. It also includes providing post-game advice, such as identifying a past decision that could have been made differently to alter the game's outcome.

\subsection{Stakeholders}
The primary stakeholders for this project are:
\begin{itemize}
    \item Players of Settlers of Catan: The end-users who will use the AI for in-game and post-game analysis to improve their skills, or train against various skilled AI bots.
    \item Dr. Istvan David: The project supervisor who provides guidance, expertise, and oversight.
    \item The Project Team: The developers responsible for designing, implementing, and testing the software.
    \item The Department of Computing and Software (CAS) at McMaster University: The organization hosting the project, which benefits from the academic and technical achievements of its students and faculty.
\end{itemize}

\subsection{Environment}
The project requires a hardware and software environment to support the various modules. The core of the system will run on a central processing unit (for example, a server or a high-powered laptop) capable of handling the computational demands of AI. 

On the software side, the system will need:
\begin{itemize}
    \item a game simulator to serve as a training environment for the RL agents
    \item a reinforcement learning framework (for example, Farama Foundation's Gymnasium) to train the AI
    \item a computer vision library (for example, YOLO) to process video feeds and translate them into a digital game state representation
    \item a user-facing application for the player, for example a website or mobile application 
    \item a visual display (for example, a virtual board or text) to show a digital representation of the game state and the AI's actions
\end{itemize}

On the hardware side, the project requires:
\begin{itemize}
    \item a computer with a processor and GPU to run the AI and computer vision modules
    \item  devices (for example, laptops or smartphones) for the players to use the application and stream video of the game board
    \item a physical Catan board game for testing the computer vision and digital twin functionalities
\end{itemize}

\section{Goals}

\section{Stretch Goals}

\section{Extras}

\wss{For CAS 741: State whether the project is a research project. This
designation, with the approval (or request) of the instructor, can be modified
over the course of the term.}

\wss{For SE Capstone: List your extras.  Potential extras include usability
testing, code walkthroughs, user documentation, formal proof, GenderMag
personas, Design Thinking, etc.  (The full list is on the course outline and in
Lecture 02.) Normally the number of extras will be two.  Approval of the extras
will be part of the discussion with the instructor for approving the project.
The extras, with the approval (or request) of the instructor, can be modified
over the course of the term.}

\newpage{}

\section*{Appendix --- Reflection}

\wss{Not required for CAS 741}

The purpose of reflection questions is to give you a chance to assess your own
learning and that of your group as a whole, and to find ways to improve in the
future. Reflection is an important part of the learning process.  Reflection is
also an essential component of a successful software development process.  

Reflections are most interesting and useful when they're honest, even if the
stories they tell are imperfect. You will be marked based on your depth of
thought and analysis, and not based on the content of the reflections
themselves. Thus, for full marks we encourage you to answer openly and honestly
and to avoid simply writing ``what you think the evaluator wants to hear.''

Please answer the following questions.  Some questions can be answered on the
team level, but where appropriate, each team member should write their own
response:


\begin{enumerate}
    \item What went well while writing this deliverable? 
    \item What pain points did you experience during this deliverable, and how
    did you resolve them?
    \item How did you and your team adjust the scope of your goals to ensure
    they are suitable for a Capstone project (not overly ambitious but also of
    appropriate complexity for a senior design project)?
\end{enumerate}  

\end{document}
