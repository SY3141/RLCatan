\documentclass{article}


\usepackage{tabularx}
\usepackage{booktabs}

\title{Software Requirements Specification for RLCatan\\\progname}

\author{\authname}

\date{}

%% Comments

\usepackage{color}

\newif\ifcomments\commentstrue %displays comments
%\newif\ifcomments\commentsfalse %so that comments do not display

\ifcomments
\newcommand{\authornote}[3]{\textcolor{#1}{[#3 ---#2]}}
\newcommand{\todo}[1]{\textcolor{red}{[TODO: #1]}}
\else
\newcommand{\authornote}[3]{}
\newcommand{\todo}[1]{}
\fi

\newcommand{\wss}[1]{\authornote{magenta}{SS}{#1}} 
\newcommand{\plt}[1]{\authornote{cyan}{TPLT}{#1}} %For explanation of the template
\newcommand{\an}[1]{\authornote{cyan}{Author}{#1}}

%% Common Parts

\newcommand{\progname}{Software Engineering} % PUT YOUR PROGRAM NAME HERE
\newcommand{\authname}{Team 8, RLCatan
\\ Rebecca Di Filippo
\\ Jake Read
\\ Matthew Cheung
\\ Sunny Yao} % AUTHOR NAMES

\usepackage{hyperref}
    \hypersetup{colorlinks=true, linkcolor=blue, citecolor=blue, filecolor=blue,
                urlcolor=blue, unicode=false}
    \urlstyle{same}
                                


\begin{document}

\maketitle

\begin{table}[hp]
\caption{Revision History} \label{TblRevisionHistory}
\begin{tabularx}{\textwidth}{llX}
\toprule
\textbf{Date} & \textbf{Developer(s)} & \textbf{Change}\\
\midrule
... & ... & ...\\
\bottomrule
\end{tabularx}
\end{table}

\section*{(G) Goals}\label{sec:srs-goals}
\renewcommand{\thesubsection}{G.\arabic{subsection}}

\subsection{Context and Overall Objective}\label{subsec:context-and-overall-objective}

\subsection{Current Situation}\label{subsec:current-situation}

\subsection{Expected Benefits}\label{subsec:expected-benefits}

\subsection{Functionality Overview}\label{subsec:functionality-overview}

\subsection{High-Level Usage Scenarios}\label{subsec:high-level-usage-scenarios}

\subsection{Limitations and Exclusions}\label{subsec:limitations-and-exclusions}

\subsection{Stakeholders and Requirements Sources}\label{subsec:stakeholders-and-requirements-sources}

\newpage{}


\section*{(E) Environment}\label{sec:srs-environment}
\renewcommand{\thesubsection}{E.\arabic{subsection}}
\setcounter{subsection}{0}

\subsection{Glossary}\label{subsec:glossary}
\begin{itemize}
    \item \textbf{Catan} – A strategy board game called \textit{Settlers of Catan} where players collect resources, build roads/settlements, and trade to earn points.
    \item \textbf{AI (Artificial Intelligence)} – Field of computer science and engineering focused on creating systems capable of performing tasks that usually require human intelligence.
    \item \textbf{RL (Reinforcement Learning)} – A type of machine learning where an agent learns how to make decisions by interacting with an environment and receiving rewards or penalties for actions taken.
    \item \textbf{Digital twin} – A digital system that mirrors a physical one. In this project, it refers to a digital representation of the physical Catan game board, updated in real time.
    \item \textbf{CV (Computer Vision)} – Enables a system to interpret and process visual data. In this project, it refers to the camera system that understands and tracks the physical Catan board state.
    \item \textbf{LLM (Large Language Model)} – A machine learning model trained on vast amounts of text data to generate and understand natural language.
    \item \textbf{Game state} – The current configuration of the game, including player resources, board layout (roads, settlements, cities), dice rolls, and ongoing trades.
\end{itemize}

\subsection{Components}\label{subsec:components2}
\begin{itemize}
    \item \textbf{Physical Catan board and pieces} – The physical game setup that the system observes.
    \item \textbf{Players} – Human participants who interact with the physical game and receive decision support.
    \item \textbf{Cameras/Sensors} – Hardware that captures the physical board state for the digital twin.
    \item \textbf{OpenAI Gym} – Provides a training and testing environment for reinforcement learning agents.
    \item \textbf{Reinforcement Learning Agent} – Learns strategies through the simulator and connects with the digital twin for real-time decision support.
    \item \textbf{Visualization Interface} – Displays the current game state and AI recommendations to users.
    \item \textbf{Optional components:}
    \begin{itemize}
        \item \textbf{Smart glasses} – Provide players with an augmented view of the game and recommendations.
        \item \textbf{LLM service/API} – Generates natural language explanations of strategies.
    \end{itemize}
\end{itemize}

\subsection{Constraints}\label{subsec:constraints}
\begin{itemize}
    \item \textbf{Game rules of Catan} – The RL agent must strictly follow the official rules of the game.
    \item \textbf{Real-time operation} – The system must process board states and provide recommendations fast enough to be useful during live play.
    \item \textbf{Camera limits (Computer Vision)} – Accuracy of game state detection is restricted by available hardware (camera resolution, field of view, lighting).
    \item \textbf{Simulator environment} – The RL agent is limited to the APIs and mechanics provided by the Catan simulator.
    \item \textbf{Computational resources} – Training and running the RL agent are bounded by available GPU/CPU capacity.
    \item \textbf{Timeframe} – The project must be completed within the allocated time frame.
\end{itemize}

\subsection{Assumptions}\label{subsec:assumptions}
\begin{itemize}
    \item \textbf{Players will follow standard Catan rules} – It is assumed that human players won’t cheat or make illegal moves.
    \item \textbf{Stable lighting and camera angle} – The computer vision module assumes it can reliably see the board under stable conditions.
    \item \textbf{Network and device reliability} – It is assumed that player devices and connections work well enough for real-time suggestions.
\end{itemize}

\subsection{Effects}\label{subsec:effects}
\begin{itemize}
    \item \textbf{Player decision support} – The AI provides move suggestions, affecting player decisions in real time.
    \item \textbf{Learning and adaptation} – The RL agent improves over time, affecting the challenge and quality of advice given to players.
    \item \textbf{Post-game analysis} – Feedback and suggestions may influence how players approach future games.
    \item \textbf{Device usage} – The system utilizes computational resources on player devices or servers for inference and visualization.
    \item \textbf{Game pacing} – Real-time suggestions could alter the flow and pacing of the game.
\end{itemize}

\subsection{Invariants}\label{subsec:invariants}
\begin{itemize}
    \item \textbf{Board orientation is fixed} – The physical board remains stable, allowing accurate computer vision tracking.
    \item \textbf{Player order remains constant} – The sequence of turns does not change unexpectedly.
    \item \textbf{Player set is fixed} – No new players join or leave mid-game.
    \item \textbf{Game components stay in place} – Settlements, roads, and resources are only moved through valid game actions.
    \item \textbf{Camera/viewing angle remains stable} – The vision system continuously views the board from a consistent perspective.
\end{itemize}

\newpage{}



\section*{(S) System}\label{sec:srs-system}
\renewcommand{\thesubsection}{S.\arabic{subsection}}
\setcounter{subsection}{0}

\subsection{Components}\label{subsec:components}
The system is separated into six major components:

\begin{itemize}
    \item Board Representation: The training
    environment for the model, acting as a representation of the game’s state and the
    moves that can be made on a given turn.

    \item Game state digital twin: The component
    in charge of transferring a static/real-time physical board state to the board
    representation, via CV or sensors.

    \item AI model: The pre-trained model,
    responsible for providing a move prompt to be sent to the user.

    \item User Interface: The visual
    representation of the current game state, appended with the move suggested by
    the model, displayed to the user so they can make use of the model’s advice.

    \item Game State Database: A database for
    storing each game state in a given game, for traceability and for providing
    context to an LLM for post-game review.

    \item Communication Layer: The component
    in charge of communication between various components, such as sending the read
    board state to the board representation or transferring the model’s provided
    move to the UI\@.
\end{itemize}

\subsection{Functionality}\label{subsec:functionality}

\subsection{Interfaces}\label{subsec:interfaces}

\subsection{Detailed Usage Scenarios}\label{subsec:detailed-usage-scenarios}

\subsection{Prioritization}\label{subsec:prioritization}

\subsection{Verification and Acceptance Criteria}\label{subsec:verification-and-acceptance-criteria}

\subsection{Requirements Process and Report}\label{subsec:requirements-process-and-report}

\newpage{}


\section*{(P) Project}\label{sec:srs_project}
\renewcommand{\thesubsection}{P.\arabic{subsection}}
\setcounter{subsection}{0}

\subsection{Roles and Personel}

\subsection{Imposed Technical Choices}

\subsection{Schedule and Milestones}

\subsection{Tasks and Deliverables}

\subsection{Requiremed Technology Elements}

\subsection{Risk and Mitigation Analysis}

\subsection{Requirements Process and Report}

\newpage{}


\section*{Appendix --- Reflection}
\begin{enumerate}
  \item What went well while writing this deliverable? 
  \item What pain points did you experience during this deliverable, and how did
  you resolve them?
  \item How many of your requirements were inspired by speaking to your
  client(s) or their proxies (e.g. your peers, stakeholders, potential users)?
  \item Which of the courses you have taken, or are currently taking, will help
  your team to be successful with your capstone project.
  \item What knowledge and skills will the team collectively need to acquire to
  successfully complete this capstone project?  Examples of possible knowledge
  to acquire include domain specific knowledge from the domain of your
  application, or software engineering knowledge, mechatronics knowledge or
  computer science knowledge.  Skills may be related to technology, or writing,
  or presentation, or team management, etc.  You should look to identify at
  least one item for each team member.
  \item For each of the knowledge areas and skills identified in the previous
  question, what are at least two approaches to acquiring the knowledge or
  mastering the skill?  Of the identified approaches, which will each team
  member pursue, and why did they make this choice?
\end{enumerate}


\subsection*{Jake Read}\label{subsec:jake-read}
\begin{enumerate}
    \item I primarily worked on the hazard analysis doc, so for this one I handled a lot of the reviewing.
    I read through each section and made comments on things that could be improved or added, and I think it went quite smoothly.
    My team members are all still pulling their weight, and everyone accepts feedback well, which is great.
    I would say the doc is more coherent now, hopefully that comes across to you as well :)

    \item The main pain point was the good old Meyer template.
    To be honest, I don't enjoy SRS documents, so I wasn't particularly thrilled to have to write another one after 3RA3.
    We were recommended to use Meyer over Volere, and I'm happy with the choice, since Volere is pointlessly long, but Meyer has its own challenges.
    For one, the template is only available in asciidoctor (why?), so I had to rewrite the whole template in LaTeX\@.
    On top of that, it definitely feels like the SRS rubric is focused more so on Volere, so making sure this one followed it too was a bit of a hassle.
    Additionally, last time we used this template, it was the final project for an entire course, but naturally there wasn't as much time for this deliverable, so it feels comparitively limited.
    We compromised my shortening certain sections, while tweaking the template in areas to add more lists of actual requirements.

    \item Quite a lot of our requirements came from chatting with Prof. Istvan David, our supervisor.
    I asked him a lot of questions regarding his expectations when proposing the project, as well as asking for clarification for more technical details we weren't aware of.
    His initial project description was also an excellent source for requirements elicitation since it was well-structured and separated rather modularly.
    Sunny also has connections in the \emph{Catan} competitive community, so we were able to derive some requirements directly from our potential user base.

    \item Most of the courses we've taken throughout uni will help obviously, since that's the fundamental idea behind a capstone project, but there are a few that are particularly relevant.
    3RA3 covered a lot of the documentation that we also have to do for this class.
    Having done similar work before definitely helps reduce some of the frustration of seemingly endless written deliverables (marginally), so I'm grateful for that.
    I'm also currently taking 4AL3 as an elective, which focuses on the application of machine learning.
    It covers reinforcement learning, which is the focus of our project, so it's a very relevant course.
    Overall though, I think the most helpful course for this project, (and in my opinion the most useful course in this degree so far), was 2AA4.
    That course was essentially capstone-lite, with one large main full-term coding project that progressed in stages, from an MVP up to a finished product.
    It also covered design principles and gave critical experience regarding the organization of group coding projects.
    It had a huge workload and was very stressful at the time, but in retrospect it was the most fun I've had in any course, and it's a big part of why I'm so excited for this project.

    \item The main skills we're going to need revolve around deep reinforcement learning and deep neural networks.
    This is the main reason I wanted to take this project in the first place.
    DRL is a relatively new field, with a lot of active research, so an opportunity to work with it to this scale is rare and develops highly marketable software engineering skills.
    With \emph{Catan}, a big area of knowledge we'll need is the creation and application of heuristics to guide the DRL model to improved results.
    I'm rather interested in this, and I'm happy to research it on my own time to try to gain some expertise for the group.

    \item For getting an understanding of DRL and DNN, we have two main ways of gaining the necessary knowledge.
    One of the options for learning this is researching on our own time, using a list of papers on the subject we're slowly collecting on our Discord.
    The other option is discussions with Istvan, since he's an expert on the subject and is happy to help with any confusions we have.
    The approach to learning heuristics to aid the model are similar, research and consulting Istvan, but a bit of trial and error will be involved too.
    By definition, no one knows exactly why heuristics work, they're a matter of intuition, so by just experimenting with different approaches I should be able to learn quite a lot.
    I'm planning on learning primarily by chatting with Istvan.
    I've already been asking him a lot of questions and his answers have been very helpful, so while he's happy to keep helping I'm happy to keep learning from him.
    If I feel like I'm taking too much of his time, I'll fall back on research and experimentation.

\end{enumerate}

\end{document}
