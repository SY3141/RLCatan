\documentclass{article}


\usepackage{tabularx}
\usepackage{booktabs}

\title{Software Requirements Specification for RLCatan\\\progname}

\author{\authname}

\date{}

\input{../Comments}
%% Common Parts

\newcommand{\progname}{Software Engineering} % PUT YOUR PROGRAM NAME HERE
\newcommand{\authname}{Team 8, RLCatan
\\ Rebecca Di Filippo
\\ Jake Read
\\ Matthew Cheung
\\ Sunny Yao} % AUTHOR NAMES

\usepackage{hyperref}
    \hypersetup{colorlinks=true, linkcolor=blue, citecolor=blue, filecolor=blue,
                urlcolor=blue, unicode=false}
    \urlstyle{same}
                                


\begin{document}

\maketitle

\begin{table}[hp]
\caption{Revision History} \label{TblRevisionHistory}
\begin{tabularx}{\textwidth}{llX}
\toprule
\textbf{Date} & \textbf{Developer(s)} & \textbf{Change}\\
\midrule
... & ... & ...\\
\bottomrule
\end{tabularx}
\end{table}

\section*{(G) Goals}\label{sec:srs-goals}
\renewcommand{\thesubsection}{G.\arabic{subsection}}

\subsection{Context and Overall Objective}\label{subsec:context-and-overall-objective}

\subsection{Current Situation}\label{subsec:current-situation}

\subsection{Expected Benefits}\label{subsec:expected-benefits}

\subsection{Functionality Overview}\label{subsec:functionality-overview}

\subsection{High-Level Usage Scenarios}\label{subsec:high-level-usage-scenarios}

\subsection{Limitations and Exclusions}\label{subsec:limitations-and-exclusions}

\subsection{Stakeholders and Requirements Sources}\label{subsec:stakeholders-and-requirements-sources}

\newpage{}


\section*{(E) Environment}\label{sec:srs-environment}
\renewcommand{\thesubsection}{E.\arabic{subsection}}
\setcounter{subsection}{0}

\subsection{Glossary}\label{subsec:glossary}

\subsection{Components}\label{subsec:components2}

\subsection{Constraints}\label{subsec:constraints}

\subsection{Assumptions}\label{subsec:assumptions}

\subsection{Effects}\label{subsec:effects}

\subsection{Invariants}\label{subsec:invariants}

\newpage{}


\section*{(S) System}\label{sec:srs-system}
\renewcommand{\thesubsection}{S.\arabic{subsection}}
\setcounter{subsection}{0}

\subsection{Components}\label{subsec:components}
The system is separated into six major components:

\begin{itemize}
    \item Board Representation: The training
    environment for the model, acting as a representation of the game’s state and the
    moves that can be made on a given turn.

    \item Game state digital twin: The component
    in charge of transferring a static/real-time physical board state to the board
    representation, via CV or sensors.

    \item AI model: The pre-trained model,
    responsible for providing a move prompt to be sent to the user.

    \item User Interface: The visual
    representation of the current game state, appended with the move suggested by
    the model, displayed to the user so they can make use of the model’s advice.

    \item Game State Database: A database for
    storing each game state in a given game, for traceability and for providing
    context to an LLM for post-game review.

    \item Communication Layer: The component
    in charge of communication between various components, such as sending the read
    board state to the board representation or transferring the model’s provided
    move to the UI\@.
\end{itemize}

\subsection{Functionality}\label{subsec:functionality}

\subsection{Interfaces}\label{subsec:interfaces}

\subsection{Detailed Usage Scenarios}\label{subsec:detailed-usage-scenarios}

\subsection{Prioritization}\label{subsec:prioritization}

\subsection{Verification and Acceptance Criteria}\label{subsec:verification-and-acceptance-criteria}

\subsection{Requirements Process and Report}\label{subsec:requirements-process-and-report}

\newpage{}


\section*{(P) Project}\label{sec:srs_project}
\renewcommand{\thesubsection}{P.\arabic{subsection}}
\setcounter{subsection}{0}

\subsection{Roles and Personel}

\subsection{Imposed Technical Choices}

\subsection{Schedule and Milestones}

\subsection{Tasks and Deliverables}

\subsection{Requiremed Technology Elements}

\subsection{Risk and Mitigation Analysis}

\subsection{Requirements Process and Report}

\newpage{}


\section*{Appendix --- Reflection}
\input{../SRS_Reflection.tex}

\end{document}
