\documentclass{article}

\usepackage{booktabs}
\usepackage{tabularx}

\title{Development Plan\\\progname}

\author{\authname}

\date{}

%% Comments

\usepackage{color}

\newif\ifcomments\commentstrue %displays comments
%\newif\ifcomments\commentsfalse %so that comments do not display

\ifcomments
\newcommand{\authornote}[3]{\textcolor{#1}{[#3 ---#2]}}
\newcommand{\todo}[1]{\textcolor{red}{[TODO: #1]}}
\else
\newcommand{\authornote}[3]{}
\newcommand{\todo}[1]{}
\fi

\newcommand{\wss}[1]{\authornote{magenta}{SS}{#1}} 
\newcommand{\plt}[1]{\authornote{cyan}{TPLT}{#1}} %For explanation of the template
\newcommand{\an}[1]{\authornote{cyan}{Author}{#1}}

%% Common Parts

\newcommand{\progname}{Software Engineering} % PUT YOUR PROGRAM NAME HERE
\newcommand{\authname}{Team 8, RLCatan
\\ Rebecca Di Filippo
\\ Jake Read
\\ Matthew Cheung
\\ Sunny Yao} % AUTHOR NAMES

\usepackage{hyperref}
    \hypersetup{colorlinks=true, linkcolor=blue, citecolor=blue, filecolor=blue,
                urlcolor=blue, unicode=false}
    \urlstyle{same}
                                


\begin{document}

\maketitle

\begin{table}[hp]
\caption{Revision History} \label{TblRevisionHistory}
\begin{tabularx}{\textwidth}{llX}
\toprule
\textbf{Date} & \textbf{Developer(s)} & \textbf{Change}\\
\midrule
Sept 17 & Rebecca Di Filippo & Draft of 1,2,5,6,7,Team Charter\\
Sept 21 & Rebecca Di Filippo & Revision of 1,2,5,6,7,Team Charter\\
Sept 21 & Rebecca Di Filippo & Addition of individual Reflection\\
... & ... & ...\\
\bottomrule
\end{tabularx}
\end{table}

\newpage{}

\wss{Put your introductory blurb here.  Often the blurb is a brief roadmap of
what is contained in the report.}

\wss{Additional information on the development plan can be found in the
\href{https://gitlab.cas.mcmaster.ca/courses/capstone/-/blob/main/Lectures/L02b_POCAndDevPlan/POCAndDevPlan.pdf?ref_type=heads}
{lecture slides}.}

\section{Confidential Information}

\raggedright
There is no confidential information to protect in this project.

\section{IP to Protect}

\raggedright
There is no IP to protect in this project.

\section{Copyright License}

\raggedright{AlgoCatan is adopting the MIT License, which can be found at \href{https://github.com/SY3141/RLCatan/blob/main/LICENSE}{this link}.}

\section{Team Meeting Plan}

\raggedright
We will schedule a meeting every time we have an addressable 
agenda. Virtual meetings will be held through a Discord call 
and physical meetings will be held at tutorial locations or ETB.
 Meetings may be hybrid depending on schedules. We will schedule
  meetings with our advisor when we require resources or 
  expertise. Meetings will be structured with a set agenda 
  before scheduling to address all of the points 
  by the liaison.


\raggedright
We will be using Discord as our main communication 
platform. We will create a server with channels for general 
discussion, meeting scheduling, and project management.
For more formal communication, we will use email to contact 
our advisor.
We will also use GitHub Issues for tracking existing 
issues and their closures.

\section{Team Member Roles}

\raggedright
\begin{itemize}
  \item Team Leader: this person is responsible for scheduling meetings,
    ensuring deadlines are met, and overall team coordination. This person will
    also be the main point of contact with the supervisor and the TA
  \item Notetaker: this person is responsible for creating meeting agendas and also
    taking notes during meetings. This person will also be updating the Kaban
    board.
  \item IT: this person is responisble for managing the GitHub repository, including
    branches, pull requests, and issues. This person will also be resonsible for trouble
    shooting any technical issues that arise.
  \item Reviewer: this person is responsible for reviewing code and documentation
    to ensure quality and consistency. This person will also be responsible for
    ensuring that coding standards are being followed.
\end{itemize}


\section{Workflow Plan}

\begin{itemize}
	\item How will you be using git, including branches, pull request, etc.?
	\item How will you be managing issues, including template issues, issue
	classification, etc.?
  \item Use of CI/CD
\end{itemize}

\raggedright
We will be using GitHub for version control and collaboration.
To manage development, we will create branches for each main feature,
along with sub-branches for individual team members’ work. Pull requests 
will be used to review and merge code changes, ensuring quality and 
consistency. GitHub Issues will be used to track tasks and bugs, with 
issues assigned to specific team members or whoever is available.
To streamline this process, we will also use issue templates for 
consistency in reporting and classification of issues. Finally, GitHub 
Actions will be used for CI/CD to automate testing and deployment.

\section{Project Decomposition and Scheduling}

\begin{itemize}
  \item How will you be using GitHub projects?
  \item Include a link to your GitHub project
\end{itemize}

\wss{How will the project be scheduled?  This is the big picture schedule, not
details. You will need to reproduce information that is in the course outline
for deadlines.}

\raggedright
We will be using GitHub Projects to manage our project tasks and milestones. We
will create a project board with columns for "To Do", "In Progress", "In Review
", and "Done". Each task will be represented as a card on the board. These tasks
will be prioritized based on their importance and deadlines. Every deadline will
driectly correspond to deadlines from the course outline.
The link to our GitHub project is *INSERT LINK HERE*


\section{Proof of Concept Demonstration Plan}

What is the main risk, or risks, for the success of your project?  What will you
demonstrate during your proof of concept demonstration to convince yourself that
you will be able to overcome this risk?

\medskip

\raggedright
To mitigate these risks, we will conduct regular Elo testing 
against existing benchmark opponents to make sure the AI is improving or at least not regressing.
To mitigate these risks, we will conduct regular elo testing 
against existing benchmark opponents to make sure the AI is improving or at least not regressing.
We will also conduct user testing sessions to gather feedback and 
make iterative improvements to the interface. 
We will also ensure that our AI models are thoroughly tested and 
validated before deployment.

\section{Expected Technology}

\raggedright
We expect to use Python as our primary programming 
language, as well as using the Catanatron open source 
libary for simulating Catan games and training our models.
We will also utilize React and JavaScript as 
our web framework for building the user interface and 
digital twin. For computer vision models, we will be using
openCV and yolov9 for image recognition and processing.


\wss{The implementation decisions can, and likely will, change over the course
of the project.  The initial documentation should be written in an abstract way;
it should be agnostic of the implementation choices, unless the implementation
choices are project constraints.  However, recording our initial thoughts on
implementation helps understand the challenge level and feasibility of a
project.  It may also help with early identification of areas where project
members will need to augment their training.}

Topics to discuss include the following:

\begin{itemize}
\item Specific programming language
\item Specific libraries
\item Pre-trained models
\item Specific linter tool (if appropriate)
\item Specific unit testing framework
\item Investigation of code coverage measuring tools
\item Specific plans for Continuous Integration (CI), or an explanation that CI
  is not being done
\item Specific performance measuring tools (like Valgrind), if
  appropriate
\item Tools you will likely be using?
\end{itemize}

\wss{git, GitHub and GitHub projects should be part of your technology.}

\section{Coding Standard}

\wss{What coding standard will you adopt?}

\newpage{}

\section*{Appendix --- Reflection}

\wss{Not required for CAS 741}

The purpose of reflection questions is to give you a chance to assess your own
learning and that of your group as a whole, and to find ways to improve in the
future. Reflection is an important part of the learning process.  Reflection is
also an essential component of a successful software development process.  

Reflections are most interesting and useful when they're honest, even if the
stories they tell are imperfect. You will be marked based on your depth of
thought and analysis, and not based on the content of the reflections
themselves. Thus, for full marks we encourage you to answer openly and honestly
and to avoid simply writing ``what you think the evaluator wants to hear.''

Please answer the following questions.  Some questions can be answered on the
team level, but where appropriate, each team member should write their own
response:


\begin{enumerate}
    \item Why is it important to create a development plan prior to starting the
    project?
    \item In your opinion, what are the advantages and disadvantages of using
    CI/CD?
    \item What disagreements did your group have in this deliverable, if any,
    and how did you resolve them?
\end{enumerate}

\newpage{}

\section*{Appendix --- Team Charter}

\wss{borrows from
\href{https://engineering.up.edu/industry_partnerships/files/team-charter.pdf}
{University of Portland Team Charter}}

\subsection*{External Goals}

\wss{What are your team's external goals for this project? These are not the
goals related to the functionality or quality of the project.  These are the
goals on what the team wishes to achieve with the project.  Potential goals are
to win a prize at the Capstone EXPO, or to have something to talk about in
interviews, or to get an A+, etc.}

\raggedright
Our teams external goals for this project are to create a have a strong understanding
of reinforcement learning and computer vision, to be able to build a hireable portfolio. 
We also want to get a A/A+ in the course, since we are putting

\subsection*{Attendance}

\subsubsection*{Expectations}

\wss{What are your team's expectations regarding meeting attendance (being on
time, leaving early, missing meetings, etc.)?}

\raggedright
Our team meets every week at 4:30-6:30 pm on Thursday. All other meetings are 
scheduled on a weekly basis, depending on deadlines and amount of work. It is expected that all
team members attend every meeting and arrive on time. If a team member is unable
to attend a meeting they must notify the team at least 24 hours in advance. If a 
team member is going to be late or must leave early they must notify the team prior
to the meeting. 

\subsubsection*{Acceptable Excuse}

\wss{What constitutes an acceptable excuse for missing a meeting or a deadline?
What types of excuses will not be considered acceptable?}

\raggedright
An acceptable excuse for missing a meeting or deadline includes sickness, family
emergencies, or academic obligations. On the other hand, unacceptable excuses include
forgetting, being lazy, or not prioritizing the team. Again, it is expected that the team
knows 24 hours in advance if a member is going to miss a meeting or deadline.

\subsubsection*{In Case of Emergency}

\wss{What process will team members follow if they have an emergency and cannot
attend a team meeting or complete their individual work promised for a team
deliverable?}

\raggedright
In case of an emergency, the team member must notify the team as soon as possible,
preferably before the meeting or deadline. If a deadline cannot be met other members
will distrubute the work among themselves to ensure the deadline is met. Its important
to note that 1 missed deadline with little notice is acceptable, but repeated offenses
will not be tolerated.

\subsection*{Accountability and Teamwork}

\subsubsection*{Quality} 

\wss{What are your team's expectations regarding the quality
of team members' preparation for team meetings and the quality of the
deliverables that members bring to the team?}

\subsubsection*{Attitude}

\wss{What are your team's expectations regarding team members' ideas,
interactions with the team, cooperation, attitudes, and anything else regarding
team member contributions?  Do you want to introduce a code of conduct?  Do you
want a conflict resolution plan?  Can adopt existing codes of conduct.}

\subsubsection*{Stay on Track}

\wss{What methods will be used to keep the team on track? How will your team
ensure that members contribute as expected to the team and that the team
performs as expected? How will your team reward members who do well and manage
members whose performance is below expectations?  What are the consequences for
someone not contributing their fair share?}

\wss{You may wish to use the project management metrics collected for the TA and
instructor for this.}

\wss{You can set target metrics for attendance, commits, etc.  What are the
consequences if someone doesn't hit their targets?  Do they need to bring the
coffee to the next team meeting?  Does the team need to make an appointment with
their TA, or the instructor?  Are there incentives for reaching targets early?}

\subsubsection*{Team Building}

\wss{How will you build team cohesion (fun time, group rituals, etc.)? }

\subsubsection*{Decision Making} 

\wss{How will you make decisions in your group? Consensus?  Vote? How will you
handle disagreements? }

\end{document}