\documentclass{article}

\usepackage{booktabs}
\usepackage{tabularx}

\title{Development Plan\\\progname}

\author{\authname}

\date{}

%% Comments

\usepackage{color}

\newif\ifcomments\commentstrue %displays comments
%\newif\ifcomments\commentsfalse %so that comments do not display

\ifcomments
\newcommand{\authornote}[3]{\textcolor{#1}{[#3 ---#2]}}
\newcommand{\todo}[1]{\textcolor{red}{[TODO: #1]}}
\else
\newcommand{\authornote}[3]{}
\newcommand{\todo}[1]{}
\fi

\newcommand{\wss}[1]{\authornote{magenta}{SS}{#1}} 
\newcommand{\plt}[1]{\authornote{cyan}{TPLT}{#1}} %For explanation of the template
\newcommand{\an}[1]{\authornote{cyan}{Author}{#1}}

%% Common Parts

\newcommand{\progname}{Software Engineering} % PUT YOUR PROGRAM NAME HERE
\newcommand{\authname}{Team 8, RLCatan
\\ Rebecca Di Filippo
\\ Jake Read
\\ Matthew Cheung
\\ Sunny Yao} % AUTHOR NAMES

\usepackage{hyperref}
    \hypersetup{colorlinks=true, linkcolor=blue, citecolor=blue, filecolor=blue,
                urlcolor=blue, unicode=false}
    \urlstyle{same}
                                


\begin{document}

\maketitle

\begin{table}[hp]
\caption{Revision History} \label{TblRevisionHistory}
\begin{tabularx}{\textwidth}{llX}
\toprule
\textbf{Date} & \textbf{Developer(s)} & \textbf{Change}\\
\midrule
Sept. 22 & Sunny,Rebecca & inital dev plan created for milestone 1\\
\bottomrule
\end{tabularx}
\end{table}

\newpage{}

\raggedright
This document outlines the development plan for RLCatan, an
AI agent that learns to play the board game \emph{Catan} competitively 
using reinforcement learning. 

\section{Confidential Information}

\raggedright
There is no confidential information to protect in this project.

\section{IP to Protect}

\raggedright
There is no IP to protect in this project.

\section{Copyright License}

\raggedright{AlgoCatan is adopting the MIT License, which can be found at \href{https://github.com/SY3141/RLCatan/blob/main/LICENSE}{this link}.}



\section{Team Meeting Plan}

\raggedright
Our team regularly meets 4:30-6:30pm every Thursday.
We will schedule additional meetings every time we have an addressable 
agenda. Virtual meetings will be held through a Discord call 
and physical meetings will be held at tutorial locations or ETB.
 Meetings may be hybrid depending on schedules. We will schedule
  meetings with our advisor when we require resources or 
  expertise. Meetings will be structured with a set agenda devised by the notetaker
  before scheduling to address all of the topics we need to discuss.

\section{Team Communication Plan}

\raggedright
We will be using Discord as our main communication 
platform. We will create a server with channels for general 
discussion, meeting scheduling, and project management.
For more formal communication, we will use email to contact 
our advisor.
We will also use GitHub Issues for tracking existing 
issues and their closures.


\section{Team Member Roles}

\raggedright
\begin{itemize}
  \item Team Leader(Jake): this person is responsible for scheduling meetings,
    ensuring deadlines are met, and overall team coordination. This person will
    also be the main point of contact with the supervisor and the TA.
  \item Notetaker(Rebecca): this person is responsible for creating meeting agendas and also
    taking notes during meetings. This person will also be updating the Kanban
    board.
  \item IT(Sunny): this person is responsible for managing the GitHub repository, including
    branches, pull requests, and issues. This person will also be resonsible for trouble-shooting
    any technical issues that arise.
  \item Researcher(Matthew): this person is responsible for researching any topics that
    the team is unfamiliar with. This person will also be responsible for finding
    relevant papers and articles to help the team understand the project better.
\end{itemize}
As the project progresses, these roles may shift depending on individual strengths
and interests. If any roles are too demanding, we will consider redistributing tasks to
ensure a balanced workload.

\section{Workflow Plan}

\raggedright
We will be using GitHub for version control and collaboration.
To manage development, we will create branches for each task on the Kanban Board,
along with sub-branches for individual team members work. Pull requests 
will be used to review and merge code changes, ensuring quality and 
consistency. GitHub Issues will be used to track tasks and bugs, with 
issues assigned to specific team members or whoever is available.
To streamline this process, we will also use issue templates for 
consistency in reporting and classification of issues.  We will be
using CI for automated testing. For CD, we will develop deployment pipelines in the
event that we have access to a computing node where we can train our RL agent.

\section{Project Decomposition and Scheduling}


\raggedright
We will be using GitHub Projects to manage our project tasks and 
track milestones. Our project Kanban board will have columns for ``To Do'', ``In Progress'', ``In Review'',
and ``Done'', with each task represented as a card that moves across the columns as it progresses. Tasks
and milestones will be scheduled according to the deadlines outlined in the course, ensuring that
the team meets all deliverable requirements on time. The teams Notetaker is resonsible for keeping
this board up to date. Large tasks will be broken down into smaller,
manageable subtasks to make progress easier to track and estimate. Task assignments will be based
on each team member’s strengths, experience, and availability. To use our resources realistically,
we will estimate the time required for each task, monitor progress. If needed we will adjust the
schedule or reassign tasks.

\medskip
The link to our GitHub project board can be found \href{https://github.com/users/SY3141/projects/1}{here}.

\medskip
The major milestones for our project are as follows:
\begin{table}[h!]
  \centering
  \begin{tabular}{|l|l|}
  \hline
  \textbf{Milestone} & \textbf{Due Date} \\ \hline
  Team Formed, Project Selected & Sept. 15 \\ \hline
  Problem Statement, POC Plan, Development Plan & Sept. 22 \\ \hline
  Requirements Document \& Hazard Analysis Revision & Oct. 6 \\ \hline
  V\&V Plan Revision & Oct. 27 \\ \hline
  Design Document Revision & Nov. 10 \\ \hline
  Proof of Concept Demonstration & Nov. 17 \\ \hline
  Design Document Revision & Jan. 19 \\ \hline
  Revision 0 Demonstration & Feb. 2 \\ \hline
  V\&V Report \& Extras Revision 0 & March 9 \\ \hline
  Final Demonstration (Revision 1) & March 23 \\ \hline
  EXPO Demonstration & TBA \\ \hline
  Final Documentation (Revision 1) & April 6 \\ \hline
  
  \end{tabular}
  \caption{Project Milestones and Due Dates}
  \label{tab:project-milestones}
  \end{table}


\section{Proof of Concept Demonstration Plan}

\raggedright
The main risks for the success of our project are the AI not 
being able to effectively learn to play \emph{Catan} at a high level, 
and the user interface not being intuitive or user-friendly. 
To mitigate these risks, we will conduct regular Elo testing
against existing benchmark opponents to make sure the AI is 
improving or at least not regressing. We will also conduct user 
testing sessions to  gather feedback and make iterative improvements to the interface. 
Lastly we will ensure that our AI models are thoroughly tested and 
validated before deployment. When we demonstrate our proof of concept,
we will showcase that the AI is able to win a game of \emph{Catan}, even if
it is not performing at a high level at this stage.

\section{Expected Technology}

\raggedright
We expect to use Python as our primary programming 
language, as well as using the Catanatron open source 
libary for simulating \emph{Catan} games and training our models.
We will also utilize React and JavaScript as 
our web framework for building the user interface and 
digital twin. For computer vision models, we will be using
openCV and yolov9 for image recognition and processing. We will be using
CI for automated testing and potentially CD for deployment pipelines.


\section{Coding Standard}

Since our projects backend is primarily in Python, we will be following the PEP 8 Coding
Standard. The link to this standard can be found
\href{https://peps.python.org/pep-0008/}{here}.

\medskip

Our frontend is primarily in JavaScript and React, so we will be following the google style guide
coding standard. The link to this standard can be found \href{https://google.github.io/styleguide/jsguide.html}{here}.

\newpage{}

\section*{Appendix --- Reflection}

%The purpose of reflection questions is to give you a chance to assess your own
learning and that of your group as a whole, and to find ways to improve in the
future. Reflection is an important part of the learning process.  Reflection is
also an essential component of a successful software development process.  

Reflections are most interesting and useful when they're honest, even if the
stories they tell are imperfect. You will be marked based on your depth of
thought and analysis, and not based on the content of the reflections
themselves. Thus, for full marks we encourage you to answer openly and honestly
and to avoid simply writing ``what you think the evaluator wants to hear.''

Please answer the following questions.  Some questions can be answered on the
team level, but where appropriate, each team member should write their own
response:


\begin{enumerate}
    \item Why is it important to create a development plan prior to starting the
    project?
    \item In your opinion, what are the advantages and disadvantages of using
    CI/CD?
    \item What disagreements did your group have in this deliverable, if any,
    and how did you resolve them?
\end{enumerate}

\subsection*{Jake Read:}\label{subsec:jake-read-reflection}
\begin{enumerate}
    \item
        A development plan helps formulate our expectations for the work in the coming months.
        It allows us to pre-plan our strategies for various milestones and encourages us to look ahead into what technology we’ll be using at each design stage.
        I feel it goes hand in hand with the problem statement in regard to preventing scope creep, as it details an initial outline we planned to follow from initiation of the project.
        Early decisions regarding elements such as the coding standard we’ll use will ensure that our design philosophy remains consistent throughout the project and prevent conflicting styles.
        Additionally, establishing confidential info, IPs to protect, copyright licenses, etc. as early as possible is critical for making decisions going forward.

    \item
        The main advantage of CI/CD in my opinion is automated testing.
        Having the full test suite run on new builds can help catch bugs much earlier and leads to improved code quality.
        It’s also faster to release new builds once all the pipelines are set up, since the process is fully automated, and this can also improve deployment consistency since there’s less human error.
        As for the disadvantages, while some of us have used CI/CD in co-ops, none of us have actually built any CI/CD pipelines before, so it will take significant effort to set it all up, and we’re likely to make mistakes that will cause deployment issues.
        It’s still probably worth it though, since using it is the only way we’ll get that experience.

    \item
        We did not come to any disagreements during this deliverable.
\end{enumerate}

\subsection*{Rebecca Di Filippo:}\label{subsec:rebecca-difilippo-reflection}
\begin{enumerate}
  \item Creating a development plan before starting a project helps the team organize the scope,
   timeline, and responsibilities. It acts as a roadmap so we don’t waste time
    moving in the wrong direction. The developmet plan also makes sure everyone is on the same page 
    about attendance, meetings, roles and responsibilities. Without a plan, there is possibility of miscommunication,
    disorganization, and going in the wrong direction.
  \item The biggest advantage of CI is that it automatically runs tests when you make a commit
   or open a pull request. That way, you know right when you commit if the code is safe to merge, which helps 
   avoid integration problems and conflicts. CD makes deploying new versions faster and more reliable,
    which saves time and helps avoid mistakes. The downside to CI/CD is that it can take some effort,
    and it’s annoying when builds are slow.
  \item Our group didn’t have major disagreements. The closest thing to a disagreement was, at first we
  thought CI/CD would be too difficult and unnecessary for the project. However, after discussing it
  with Dr. Istvan David, we concluded that it is worth including given the scope of our project. Its also
  a good learning experience.
\end{enumerate}

\subsection*{Sunny Yao:}\label{subsec:sunny-yao-reflection}
\begin{enumerate}
  \item I have the opinion that initial planning is always highly important.
  It gets the team on the same page and sets expectations for the project.
  Most importantly, it helps set the vision for the project and define project scope.
  This lets members think about the technologies that will be used and the potential challenges.
  \item An advantage of CI/CD is that it automates initial testing of future
commits and pull requests. This helps catch bugs early and avoid builds crashing in production.
The main disadvantage is the maintenance overhead from the setup as well as the
time barrier needed for each pull request. This can then incentivize
developers to make large infrequent commits to reduce the number of times that
the CI/CD pipeline needs to be run.
  \item We disagreed on initial project selection but each member was given
a chance to make their case through votes and meeting discussions. This helped to
reach a consensus on project direction. In addition, we were able to meet with our
supervisor to get further motivations to pursue our final project idea.
\end{enumerate}

\subsection*{Matthew Cheung:}\label{subsec:matthew-cheung-reflection}
\begin{enumerate}
  \item Making a development plan is important because it acts as a roadmap for the entire project.
  It makes the team to sit down and explicitly think about the next steps we want to take to reach our goal.
  Without it, we would start coding without direction and end up wasting a ton of time developing the wrong things.
  The plan also helps us limit the scope of the project, setting distinct boundaries and limits to whatever work we do.

  \item The biggest advantage for CI/CD is how it automates tedious tasks in the project.
  By automatically building and testing our code it saves the team valuable time during development.
  This catches bugs earlier and ensures of no integration problems when we try to merge code, keeping the project stable.
  However, the downside is the overhead costs it takes to set up.
  It can be a lot of work to get the pipelines running, and for a small team like ours, the time could be costly.
  Additionally, if the builds are slow, it can become annoying and slow down our workflow.

  \item We had some initial disagreements for our project choice, where we were debating between two different projects, but we ended up being force to choose Catan anyway due to prof availability so it didn't really matter.
  Once we got over the initial hurdle and started breaking down the work, we all got on the same page and were ready to work.

\end{enumerate}

\section*{Appendix --- Team Charter}


\subsection*{External Goals}

%\wss{What are your team's external goals for this project? These are not the
%goals related to the functionality or quality of the project.  These are the
%goals on what the team wishes to achieve with the project.  Potential goals are
%to win a prize at the Capstone EXPO, or to have something to talk about in
%interviews, or to get an A+, etc.}

\raggedright
Our teams external goals for this project are to attain a strong understanding
of reinforcement learning and computer vision, to be able to build a hireable portfolio. 
We also are aiming to get a A/A+ in the course, since we plan to put a lot of effort into the
project.

\subsection*{Attendance}

\subsubsection*{Expectations}

%\wss{What are your team's expectations regarding meeting attendance (being on
%time, leaving early, missing meetings, etc.)?}

\raggedright
Our team meets every week at 4:30-6:30 pm on Thursday. All other meetings are 
scheduled on a weekly basis, depending on deadlines and amount of work. It is expected that all
team members attend every meeting and arrive on time. If a team member is unable
to attend a meeting they must notify the team at least 12 hours in advance (withholding an emergency situation). If a 
team member is going to be late or must leave early they must notify the team prior
to the meeting. 

\subsubsection*{Acceptable Excuse}

%\wss{What constitutes an acceptable excuse for missing a meeting or a deadline?
%What types of excuses will not be considered acceptable?}

\raggedright
An acceptable excuse for missing a meeting or deadline includes sickness, family
emergencies, or academic obligations. On the other hand, unacceptable excuses include
forgetting, being lazy, or not prioritizing the team. Again, it is expected that the team
knows 12 hours in advance if a member is going to miss a meeting or deadline. This time
frame allows the team to adjust their plans accordingly.

\subsubsection*{In Case of Emergency}

%\wss{What process will team members follow if they have an emergency and cannot
%attend a team meeting or complete their individual work promised for a team
%deliverable?}

\raggedright
In case of an emergency, the team member must notify the team as soon as possible, before
the meeting or deadline. If a deadline cannot be met, other members will distribute 
the work among themselves to ensure the deadline is met. Its important to note that 
1-2 missed deadline with little notice is acceptable, but repeated offenses will not 
be tolerated.

\subsection*{Accountability and Teamwork}

\subsubsection*{Quality} 

%\wss{What are your team's expectations regarding the quality
%of team members' preparation for team meetings and the quality of the
%deliverables that members bring to the team?}

\raggedright
It is expected that all team members come to meetings prepared with the assigned
task completed for each week. It is expected that all tasks we complete are of high quality
and have been reviewed by at least one other team member before submission. Each
deliverable will be reviewed by all team members before submission to ensure quality 
and consistency. Lastly, all coding should follow the agreed upon coding standards.


\subsubsection*{Attitude}

%\wss{What are your team's expectations regarding team members' ideas,
%interactions with the team, cooperation, attitudes, and anything else regarding
%team member contributions?  Do you want to introduce a code of conduct?  Do you
%want a conflict resolution plan?  Can adopt existing codes of conduct.}

\raggedright
The team leader is responsible for ensuring that all team members are treated fairly.
All team members need to be open to new ideas and willing to compromise. If conflicts
arise, the team will discuss the issue and come to a resolution that works for everyone.
If a team member is not contributing to the team, the team leader will address the issue
with the member privately. If the issue persists, the team will discuss the issue with
the TA or instructor.

\subsubsection*{Stay on Track}

Our team will stay on track by setting clear weekly goals, 
using GitHub Kanban boards to monitor progress, and
holding regular check-ins to ensure accountability. Tasks 
will be distributed fairly according to skills and proficiencies, 
and progress will be tracked through version control commits and 
documented updates. Members who perform well will be recognized 
for their contributions and can take greater lead on project
direction in future tasks. If a member's performance 
falls below expectations, we will first address the issue 
through direct communication and support, ensuring that 
individual effort is accurately reflected in evaluations.
We expect to have high attendance from all members with prior 
warning for any absences.

\subsubsection*{Team Building}

%\wss{How will you build team cohesion (fun time, group rituals, etc.)? }

\raggedright
Considering our team will be working together for 8 months, its important to build
team cohesion. We will do this by scheduling fun activities outside of meetings. one
of those fun activities will be a team dinner at the end of the project to celebrate.
We also plan to play \textit{Catan} together outside of meetings. 

\subsubsection*{Decision Making} 

%\wss{How will you make decisions in your group? Consensus?  Vote? How will you
%handle disagreements? }

\raggedright
Our team will converse together before making any major decisions. We will try to reach
a consensus, but if we are unable to do so we will vote through discord polls. In the event
there is a disagreement , the team leader will have the final say.

\end{document}