\documentclass[12pt, titlepage]{article}

\usepackage{amsmath, mathtools}

\usepackage[round]{natbib}
\usepackage{amsfonts}
\usepackage{amssymb}
\usepackage{graphicx}
\usepackage{colortbl}
\usepackage{xr}
\usepackage{hyperref}
\usepackage{longtable}
\usepackage{xfrac}
\usepackage{tabularx}
\usepackage{float}
\usepackage{siunitx}
\usepackage{booktabs}
\usepackage{multirow}
\usepackage[section]{placeins}
\usepackage{caption}
\usepackage{fullpage}

\hypersetup{
bookmarks=true,     % show bookmarks bar?
colorlinks=true,       % false: boxed links; true: colored links
linkcolor=red,          % color of internal links (change box color with linkbordercolor)
citecolor=blue,      % color of links to bibliography
filecolor=magenta,  % color of file links
urlcolor=cyan          % color of external links
}

\usepackage{array}

\externaldocument{/C:/Users/jaker/OneDrive/Academics/University/Year 4 Courses/Capstone/RLCatan/docs/SRS}

\input{../../Comments}
%% Common Parts

\newcommand{\progname}{Software Engineering} % PUT YOUR PROGRAM NAME HERE
\newcommand{\authname}{Team 8, RLCatan
\\ Rebecca Di Filippo
\\ Jake Read
\\ Matthew Cheung
\\ Sunny Yao} % AUTHOR NAMES

\usepackage{hyperref}
    \hypersetup{colorlinks=true, linkcolor=blue, citecolor=blue, filecolor=blue,
                urlcolor=blue, unicode=false}
    \urlstyle{same}
                                


\begin{document}

\title{Module Interface Specification for \progname{}}

\author{\authname}

\date{\today}

\maketitle

\pagenumbering{roman}

\section{Revision History}

\begin{tabularx}{\textwidth}{p{3cm}p{2cm}X}
\toprule {\bf Date} & {\bf Version} & {\bf Notes}\\
\midrule
11/03/2025 & 1.0 & Draft Rev 1\\
\bottomrule
\end{tabularx}

~\newpage

\section{Symbols, Abbreviations and Acronyms}

See SRS Documentation at \wss{give url}

\wss{Also add any additional symbols, abbreviations or acronyms}

\newpage

\tableofcontents

\newpage

\pagenumbering{arabic}

\section{Introduction}

The following document details the Module Interface Specifications for
our project RLCatan. This project aims to create a competent reinforcement learning AI agent designed to master the board game Settlers of Catan through autonomous self-play training. The AI will use deep reinforcement learning algorithms to learn optimal decision-making strategies across several game states including resource management, territory expansion and adaptive responses to opponent actions.

Complementary documents include the System Requirement Specifications
and Module Guide.  The full documentation and implementation can be
found at \url{https://github.com/SY3141/RLCatan}.

\section{Notation}

\wss{You should describe your notation.  You can use what is below as
  a starting point.}

The structure of the MIS for modules comes from \citet{HoffmanAndStrooper1995},
with the addition that template modules have been adapted from
\cite{GhezziEtAl2003}.  The mathematical notation comes from Chapter 3 of
\citet{HoffmanAndStrooper1995}.  For instance, the symbol := is used for a
multiple assignment statement and conditional rules follow the form $(c_1
\Rightarrow r_1 | c_2 \Rightarrow r_2 | ... | c_n \Rightarrow r_n )$.

The following table summarizes the primitive data types used by \progname. 

\begin{center}
\renewcommand{\arraystretch}{1.2}
\noindent 
\begin{tabular}{l l p{7.5cm}} 
\toprule 
\textbf{Data Type} & \textbf{Notation} & \textbf{Description}\\ 
\midrule
character & char & a single symbol or digit\\
integer & $\mathbb{Z}$ & a number without a fractional component in (-$\infty$, $\infty$) \\
natural number & $\mathbb{N}$ & a number without a fractional component in [1, $\infty$) \\
real & $\mathbb{R}$ & any number in (-$\infty$, $\infty$)\\
\bottomrule
\end{tabular} 
\end{center}

\noindent
The specification of \progname \ uses some derived data types: sequences, strings, and
tuples. Sequences are lists filled with elements of the same data type. Strings
are sequences of characters. Tuples contain a list of values, potentially of
different types. In addition, \progname \ uses functions, which
are defined by the data types of their inputs and outputs. Local functions are
described by giving their type signature followed by their specification.

\section{Module Decomposition}

The following table is taken directly from the Module Guide document for this project.

\begin{table}[h!]
\centering
\begin{tabular}{p{0.3\textwidth} p{0.6\textwidth}}
\toprule
\textbf{Level 1} & \textbf{Level 2}\\
\midrule

\multirow{2}{0.3\textwidth}{Hardware-Hiding Module} & Hardware-Hiding Module (OS) \\
& Computer Vision Model \\
\midrule

\multirow{3}{0.3\textwidth}{Behaviour-Hiding Module} &  User Interface \\
&  Game State Manager \\
&  Reinforcement Learning Environment \\
\midrule

\multirow{3}{0.3\textwidth}{Software Decision Module} &  AI Model \\
& Game State Database \\
&  Image Queue \\
\bottomrule

\end{tabular}
\end{table}

\newpage
~\newpage

\section{MIS of \wss{Module Name}} \label{Module} \wss{Use labels for
  cross-referencing}

\wss{You can reference SRS labels, such as R\ref{R_Inputs}.}

\wss{It is also possible to use \LaTeX for hypperlinks to external documents.}

\subsection{Module}

\wss{Short name for the module}

\subsection{Uses}


\subsection{Syntax}

\subsubsection{Exported Constants}

\subsubsection{Exported Access Programs}

\begin{center}
\begin{tabular}{p{2cm} p{4cm} p{4cm} p{2cm}}
\hline
\textbf{Name} & \textbf{In} & \textbf{Out} & \textbf{Exceptions} \\
\hline
\wss{accessProg} & - & - & - \\
\hline
\end{tabular}
\end{center}

\subsection{Semantics}

\subsubsection{State Variables}

\wss{Not all modules will have state variables.  State variables give the module
  a memory.}

\subsubsection{Environment Variables}

\wss{This section is not necessary for all modules.  Its purpose is to capture
  when the module has external interaction with the environment, such as for a
  device driver, screen interface, keyboard, file, etc.}

\subsubsection{Assumptions}

\wss{Try to minimize assumptions and anticipate programmer errors via
  exceptions, but for practical purposes assumptions are sometimes appropriate.}

\subsubsection{Access Routine Semantics}

\noindent \wss{accessProg}():
\begin{itemize}
\item transition: \wss{if appropriate} 
\item output: \wss{if appropriate} 
\item exception: \wss{if appropriate} 
\end{itemize}

\wss{A module without environment variables or state variables is unlikely to
  have a state transition.  In this case a state transition can only occur if
  the module is changing the state of another module.}

\wss{Modules rarely have both a transition and an output.  In most cases you
  will have one or the other.}

\subsubsection{Local Functions}

\wss{As appropriate} \wss{These functions are for the purpose of specification.
  They are not necessarily something that is going to be implemented
  explicitly.  Even if they are implemented, they are not exported; they only
  have local scope.}

%%%%%%%%%%%%%%%%%ADDED MODULES BELOW THIS LINE%%%%%%%%%%%%%%%%%%%%%
\section{MIS of Hardware-Hiding Module} \label{M1}
\wss{Use labels for cross-referencing}

\wss{You can reference SRS labels, such as R\ref{R_Inputs}.}

\subsection{Module}
Hardware-Hiding Module (M1)

\subsection{Uses}
Provides a hardware abstraction layer for other modules. Interfaces with OS-level drivers and the camera hardware or simulation layer to provide consistent input streams.

\subsection{Syntax}

\subsubsection{Exported Constants}
\begin{itemize}
\item \texttt{CAMERA\_RESOLUTION} - resolution of captured frames
\item \texttt{FRAME\_RATE} - frames per second captured by the hardware layer
\end{itemize}

\subsubsection{Exported Access Programs}
\begin{center}
\begin{tabular}{p{5cm} p{3cm} p{3cm} p{3cm}}
\hline
\textbf{Name} & \textbf{In} & \textbf{Out} & \textbf{Exceptions} \\
\hline
initializeHardware & - & Boolean & HardwareInitError \\
captureFrame & - & FrameData & CaptureError \\
shutdownHardware & - & Boolean & HardwareShutdownError \\
\hline
\end{tabular}
\end{center}

\subsection{Semantics}

\subsubsection{State Variables}
\begin{itemize}
\item \texttt{hardwareStatus}: Boolean - indicates if hardware is initialized
\item \texttt{frameBuffer}: FrameData - stores the latest captured frame
\end{itemize}

\subsubsection{Environment Variables}
\begin{itemize}
\item \texttt{CameraDevice} - physical or virtual camera input
\item \texttt{DisplayInterface} - optional screen interface for visualization
\end{itemize}

\subsubsection{Assumptions}
The module assumes that the camera or simulated device is available and that OS drivers are functional.

\subsubsection{Access Routine Semantics}

\noindent initializeHardware():
\begin{itemize}
\item transition: sets \texttt{hardwareStatus} to True on successful initialization
\item output: returns True on success
\item exception: raises \texttt{HardwareInitError} if initialization fails
\end{itemize}

\noindent captureFrame():
\begin{itemize}
\item transition: updates \texttt{frameBuffer} with latest captured frame
\item output: returns \texttt{FrameData}
\item exception: raises \texttt{CaptureError} if frame cannot be captured
\end{itemize}

\noindent shutdownHardware():
\begin{itemize}
\item transition: sets \texttt{hardwareStatus} to False and releases resources
\item output: returns True on successful shutdown
\item exception: raises \texttt{HardwareShutdownError} if cleanup fails
\end{itemize}

\subsubsection{Local Functions}
\begin{itemize}
\item \texttt{checkDeviceConnection()}: ensures camera is available
\item \texttt{allocateBufferMemory()}: manages memory for frame storage
\item \texttt{releaseResources()}: frees hardware resources on shutdown
\end{itemize}

%%%%%%%%%%%%%%%%%%%%%%%%%%%%%%%%%%%%%%%%%%%%%%

\section{MIS of Computer Vision Module} \label{M2}

\subsection{Module}
Computer Vision Module (M2)

\subsection{Uses}
Processes raw camera input and translates it into structured board state data. Provides confidence metrics for detection and error correction.

\subsection{Syntax}

\subsubsection{Exported Constants}
\begin{itemize}
\item \texttt{DETECTION\_THRESHOLD} - minimum confidence for object recognition
\item \texttt{MODEL\_PATH} - path to trained CV model (YOLOv9/OpenCV)
\end{itemize}

\subsubsection{Exported Access Programs}
\begin{center}
\begin{tabular}{p{5cm} p{3cm} p{3cm} p{3cm}}
\hline
\textbf{Name} & \textbf{In} & \textbf{Out} & \textbf{Exceptions} \\
\hline
processFrame & FrameData & GameStateData & ProcessingError \\
detectBoardElements & FrameData & List[Elements] & DetectionError \\
getConfidenceMetrics & - & Dict & - \\
\hline
\end{tabular}
\end{center}

\subsection{Semantics}

\subsubsection{State Variables}
\begin{itemize}
\item \texttt{currentGameState}: structured representation of board state
\item \texttt{lastFrame}: last processed frame
\end{itemize}

\subsubsection{Environment Variables}
\begin{itemize}
\item Access to \texttt{Hardware-Hiding Module} frame buffer
\end{itemize}

\subsubsection{Assumptions}
Assumes that frames provided by M1 are valid and that model files are correctly loaded.

\subsubsection{Access Routine Semantics}

\noindent processFrame(frame):
\begin{itemize}
\item transition: converts frame into structured game state
\item output: returns \texttt{GameStateData}
\item exception: raises \texttt{ProcessingError} if parsing fails
\end{itemize}

\noindent detectBoardElements(frame):
\begin{itemize}
\item transition: identifies tiles, pieces, and player actions in the frame
\item output: returns list of detected elements
\item exception: raises \texttt{DetectionError} if detection fails
\end{itemize}

\noindent getConfidenceMetrics():
\begin{itemize}
\item output: returns confidence scores for last processed frame
\end{itemize}

\subsubsection{Local Functions}
\begin{itemize}
\item \texttt{calibrateCamera()}: adjusts image for lens distortion
\item \texttt{filterNoise()}: removes spurious detections
\end{itemize}

\section{MIS of User Interface Module} \label{M3}

\subsection{Module}
User Interface (M3)

\subsection{Uses}
Provides interactive visualization of board state and AI move suggestions. Allows users to correct misdetections from the CV module.

\subsection{Syntax}

\subsubsection{Exported Constants}
\begin{itemize}
\item \texttt{REFRESH\_RATE} - frequency of UI updates
\end{itemize}

\subsubsection{Exported Access Programs}
\begin{center}
\begin{tabular}{p{5cm} p{3cm} p{3cm} p{3cm}}
\hline
\textbf{Name} & \textbf{In} & \textbf{Out} & \textbf{Exceptions} \\
\hline
renderBoard & GameStateData & - & RenderError \\
displayAIMove & AIMove & - & - \\
getUserCorrections & - & List[Corrections] & - \\
\hline
\end{tabular}
\end{center}

\subsection{Semantics}

\subsubsection{State Variables}
\begin{itemize}
\item \texttt{uiState}: current UI rendering data
\end{itemize}

\subsubsection{Environment Variables}
\begin{itemize}
\item Display device or browser
\end{itemize}

\subsubsection{Assumptions}
Assumes the rendering environment is compatible with frontend framework and supports required update rate.

\subsubsection{Access Routine Semantics}

\noindent renderBoard(state):
\begin{itemize}
\item transition: updates UI elements to reflect the current game state
\item output: -
\item exception: raises \texttt{RenderError} if rendering fails
\end{itemize}

\noindent displayAIMove(move):
\begin{itemize}
\item transition: overlays AI recommended move on board
\item output: -
\item exception: -
\end{itemize}

\noindent getUserCorrections():
\begin{itemize}
\item output: returns user corrections for misdetected board state
\end{itemize}

\subsubsection{Local Functions}
\begin{itemize}
\item \texttt{updateDOM()}: handles DOM updates
\item \texttt{highlightElements()}: highlights tiles, pieces, or moves
\end{itemize}

%%%%%%%%%%%%%%%%%%%%%%%%%%%%%%%%%%%%%%%%%%%%%%

\section{MIS of Game State Manager Module} \label{M4}

\subsection{Module}
Game State Manager (M4)

\subsection{Uses}
Maintains the “Digital Twin” of the Catan board, validates moves, and enforces rules.

\subsection{Syntax}

\subsubsection{Exported Constants}
\begin{itemize}
\item \texttt{MAX\_PLAYERS} = 4
\end{itemize}

\subsubsection{Exported Access Programs}
\begin{center}
\begin{tabular}{p{5cm} p{3cm} p{3cm} p{3cm}}
\hline
\textbf{Name} & \textbf{In} & \textbf{Out} & \textbf{Exceptions} \\
\hline
updateState & MoveData & Boolean & InvalidMoveError \\
getState & - & GameStateData & - \\
validateMove & MoveData & Boolean & - \\
\hline
\end{tabular}
\end{center}

\subsection{Semantics}

\subsubsection{State Variables}
\begin{itemize}
\item \texttt{currentGameState}
\item \texttt{playerAssets}: resources, settlements, roads
\end{itemize}

\subsubsection{Environment Variables}
- None

\subsubsection{Assumptions}
Assumes moves are provided in standard MoveData format and game rules are well-defined.

\subsubsection{Access Routine Semantics}

\noindent updateState(move):
\begin{itemize}
\item transition: applies move to \texttt{currentGameState}
\item output: returns True if successful
\item exception: raises \texttt{InvalidMoveError} if move is illegal
\end{itemize}

\noindent getState():
\begin{itemize}
\item output: returns \texttt{currentGameState}
\end{itemize}

\noindent validateMove(move):
\begin{itemize}
\item output: returns True if move is valid according to game rules
\end{itemize}

\subsubsection{Local Functions}
\begin{itemize}
\item \texttt{calculateResources()}, \texttt{updateScores()}, \texttt{checkVictory()}
\end{itemize}

%%%%%%%%%%%%%%%%%%%%%%%%%%%%%%%%%%%%%%%%%%%%%%

\section{MIS of Reinforcement Learning Environment Module} \label{M5}

\subsection{Module}
Reinforcement Learning Environment (M5)

\subsection{Uses}
Simulates Catan rules and opponent behavior for AI training and evaluation.

\subsection{Syntax}

\subsubsection{Exported Constants}
\begin{itemize}
\item \texttt{MAX\_TURNS}
\item \texttt{REWARD\_SCALE}
\end{itemize}

\subsubsection{Exported Access Programs}
\begin{center}
\begin{tabular}{p{2cm} p{4cm} p{4cm} p{2cm}}
\hline
\textbf{Name} & \textbf{In} & \textbf{Out} & \textbf{Exceptions} \\
\hline
step & AIMove & GameStateData, Reward & StepError \\
reset & - & GameStateData & - \\
render & - & - & RenderError \\
\hline
\end{tabular}
\end{center}

\subsection{Semantics}

\subsubsection{State Variables}
\begin{itemize}
\item \texttt{simulatedState}
\item \texttt{turnCount}
\end{itemize}

\subsubsection{Environment Variables}
- None

\subsubsection{Assumptions}
Assumes AI actions are in valid AIMove format and that simulated rules match official Catan rules.

\subsubsection{Access Routine Semantics}

\noindent step(action):
\begin{itemize}
\item transition: applies action to \texttt{simulatedState}, increments \texttt{turnCount}
\item output: returns new state and reward
\item exception: raises \texttt{StepError} if action is invalid
\end{itemize}

\noindent reset():
\begin{itemize}
\item transition: resets environment to initial state
\item output: returns initial \texttt{simulatedState}
\end{itemize}

\noindent render():
\begin{itemize}
\item transition: visualizes \texttt{simulatedState}
\item exception: raises \texttt{RenderError} on failure
\end{itemize}

\subsubsection{Local Functions}
\begin{itemize}
\item \texttt{applyOpponentLogic()}, \texttt{calculateReward()}
\end{itemize}

\section{MIS of AI Model Module} \label{M6}

\subsection{Module}
AI Model (M6)

\subsection{Uses}
Analyzes game states to recommend optimal moves using deep reinforcement learning. Provides move confidence scores and textual reasoning.

\subsection{Syntax}

\subsubsection{Exported Constants}
\begin{itemize}
\item \texttt{MODEL\_PATH} - path to trained neural network weights
\end{itemize}

\subsubsection{Exported Access Programs}
\begin{center}
\begin{tabular}{p{5cm} p{3cm} p{3cm} p{3cm}}
\hline
\textbf{Name} & \textbf{In} & \textbf{Out} & \textbf{Exceptions} \\
\hline
predictMove & GameStateData & AIMove & PredictionError \\
getMoveConfidence & AIMove & Float & - \\
explainMove & AIMove & String & - \\
\hline
\end{tabular}
\end{center}

\subsection{Semantics}

\subsubsection{State Variables}
\begin{itemize}
\item \texttt{modelWeights} - neural network parameters
\item \texttt{policyNetwork} - DRL policy network
\end{itemize}

\subsubsection{Environment Variables}
- None

\subsubsection{Assumptions}
Assumes input game states are valid and model weights are loaded correctly.

\subsubsection{Access Routine Semantics}

\noindent predictMove(state):
\begin{itemize}
\item transition: evaluates state using policy network
\item output: returns optimal move (\texttt{AIMove})
\item exception: raises \texttt{PredictionError} if model fails
\end{itemize}

\noindent getMoveConfidence(move):
\begin{itemize}
\item output: returns confidence score of move
\end{itemize}

\noindent explainMove(move):
\begin{itemize}
\item output: returns textual reasoning for move selection
\end{itemize}

\subsubsection{Local Functions}
\begin{itemize}
\item \texttt{preprocessState()}, \texttt{postprocessOutput()}
\end{itemize}

%%%%%%%%%%%%%%%%%%%%%%%%%%%%%%%%%%%%%%%%%%%%%%

\section{MIS of Game State Database Module} \label{M7}

\subsection{Module}
Game State Database (M7)

\subsection{Uses}
Stores complete history of games and player actions. Provides access for analysis, replay, or post-game evaluation.

\subsection{Syntax}

\subsubsection{Exported Constants}
\begin{itemize}
\item \texttt{DB\_PATH} - database file or server location
\item \texttt{MAX\_ENTRIES} - maximum stored states
\end{itemize}

\subsubsection{Exported Access Programs}
\begin{center}
\begin{tabular}{p{2cm} p{4cm} p{4cm} p{2cm}}
\hline
\textbf{Name} & \textbf{In} & \textbf{Out} & \textbf{Exceptions} \\
\hline
writeState & GameStateData & Boolean & DBWriteError \\
readState & String & GameStateData & DBReadError \\
queryHistory & String & List[GameStateData] & DBReadError \\
\hline
\end{tabular}
\end{center}

\subsection{Semantics}

\subsubsection{State Variables}
\begin{itemize}
\item \texttt{dbConnection} - current database connection
\item \texttt{gameRecords} - cached records in memory
\end{itemize}

\subsubsection{Environment Variables}
- File system or database server

\subsubsection{Assumptions}
Assumes database is accessible and schema matches expected structure.

\subsubsection{Access Routine Semantics}

\noindent writeState(state):
\begin{itemize}
\item transition: stores state in database
\item output: returns True on success
\item exception: raises \texttt{DBWriteError} if write fails
\end{itemize}

\noindent readState(gameID):
\begin{itemize}
\item output: returns \texttt{GameStateData} for specified game
\item exception: raises \texttt{DBReadError} if retrieval fails
\end{itemize}

\noindent queryHistory(playerID):
\begin{itemize}
\item output: returns list of \texttt{GameStateData} for given player
\item exception: raises \texttt{DBReadError} if query fails
\end{itemize}

\subsubsection{Local Functions}
\begin{itemize}
\item \texttt{serializeState()}, \texttt{deserializeState()}
\end{itemize}

%%%%%%%%%%%%%%%%%%%%%%%%%%%%%%%%%%%%%%%%%%%%%%

\section{MIS of Image Queue Module} \label{M8}

\subsection{Module}
Image Queue (M8)

\subsection{Uses}
Provides low-latency asynchronous communication between CV, Game State Manager, AI Model, and UI modules.

\subsection{Syntax}

\subsubsection{Exported Constants}
\begin{itemize}
\item \texttt{QUEUE\_SIZE} - maximum buffer size
\item \texttt{TIMEOUT} - maximum wait time for enqueue/dequeue
\end{itemize}

\subsubsection{Exported Access Programs}
\begin{center}
\begin{tabular}{p{2cm} p{4cm} p{4cm} p{2cm}}
\hline
\textbf{Name} & \textbf{In} & \textbf{Out} & \textbf{Exceptions} \\
\hline
enqueue & FrameData & Boolean & QueueFullError \\
dequeue & - & FrameData & QueueEmptyError \\
peek & - & FrameData & - \\
\hline
\end{tabular}
\end{center}

\subsection{Semantics}

\subsubsection{State Variables}
\begin{itemize}
\item \texttt{queueBuffer} - stores frames in order
\item \texttt{head}, \texttt{tail} - pointers for buffer management
\end{itemize}

\subsubsection{Environment Variables}
- None

\subsubsection{Assumptions}
Assumes frames are correctly formatted and queue size is sufficient for intended throughput.

\subsubsection{Access Routine Semantics}

\noindent enqueue(frame):
\begin{itemize}
\item transition: adds frame to \texttt{queueBuffer}
\item output: returns True if successful
\item exception: raises \texttt{QueueFullError} if buffer is full
\end{itemize}

\noindent dequeue():
\begin{itemize}
\item transition: removes frame from \texttt{queueBuffer}
\item output: returns next \texttt{FrameData}
\item exception: raises \texttt{QueueEmptyError} if queue is empty
\end{itemize}

\noindent peek():
\begin{itemize}
\item output: returns next frame without removal
\end{itemize}

\subsubsection{Local Functions}
\begin{itemize}
\item \texttt{isEmpty()}, \texttt{isFull()}, \texttt{resetQueue()}
\end{itemize}

\newpage

\bibliographystyle {plainnat}
\bibliography {../../../refs/References}

\newpage

\section{Appendix} \label{Appendix}

\wss{Extra information if required}

\newpage{}

\section*{Appendix --- Reflection}

\wss{Not required for CAS 741 projects}

The information in this section will be used to evaluate the team members on the
graduate attribute of Problem Analysis and Design.

The purpose of reflection questions is to give you a chance to assess your own
learning and that of your group as a whole, and to find ways to improve in the
future. Reflection is an important part of the learning process.  Reflection is
also an essential component of a successful software development process.  

Reflections are most interesting and useful when they're honest, even if the
stories they tell are imperfect. You will be marked based on your depth of
thought and analysis, and not based on the content of the reflections
themselves. Thus, for full marks we encourage you to answer openly and honestly
and to avoid simply writing ``what you think the evaluator wants to hear.''

Please answer the following questions.  Some questions can be answered on the
team level, but where appropriate, each team member should write their own
response:

\subsection*{Jake Read:}\label{subsec:jake-read-reflection}
\begin{enumerate}
    \item
        Most aspects of this deliverable went quite smoothly.
        We were able to fairly split the work, in such a way that we all contributed important sections, but were able to work in our own time, reducing time spent in meetings.
        I believe we were able to balance things in such a way that despite working on different sections, each of us still has a general understanding of the content in each part.
        What I'm happiest about is that we all seem to be working well together.
        There hasn't been any conflict between the four of us, and we were upfront about our schedules and availability and have stuck to the expectations we set.
        Whenever I ran into issues or had questions, it was easy to get a hold of someone in our Discord to help.
        The Discord server I set up has been perfect for organization, we have a general chat and various channels for sources, notes, resources, etc.
        Whenever we had any questions none of us could answer, we reached out to our TA or our supervising professor, who were happy to help.

    \item
        We had a couple pain points during the project selection.
        We knew we wanted to work with AI/ML in the project, but not what project we wanted to do, so we began by running polls in our Discord server on the various potential projects.
        After voting and a subsequent meeting, we narrowed our options down to two.
        I was mostly interested in the \emph{Catan} project, while the rest of the team was less certain which they preferred.
        We decided to schedule a meeting with the supervising profs of both projects, to get a more in-depth idea of what each one involved.
        This worked, and we ended up going with the \emph{Catan} project.
        During this whole process, one teammate missed both meetings with little explanation, and was very slow to answer messages.
        After a discussion with the group, we agreed that we were concerned by the lack of communication, and decided to gracefully part ways with our fifth member.

    \item
        Decisions surrounding scope were quite complex, as none of us had extensive prior experience with reinforcement learning.
        This made it hard to judge how long certain aspects of the project would take, so we turned to our supervisor, Professor Istvan David.
        He was able to give us rough estimates regarding the scope/viability of various goals, which was a great help.
        The nature of our project made it quite simple to separate goals however, as the design process is rather modular (build simulation, train model, return data, etc.).
        The existing project description provided in the potential projects document also helped in this regard, as certain milestones were already marked as optional, making them clear contenders for stretch goals.

\end{enumerate}


\begin{enumerate}
  \item What went well while writing this deliverable? 
  \item What pain points did you experience during this deliverable, and how
    did you resolve them?
  \item Which of your design decisions stemmed from speaking to your client(s)
  or a proxy (e.g. your peers, stakeholders, potential users)? For those that
  were not, why, and where did they come from?
  \item While creating the design doc, what parts of your other documents (e.g.
  requirements, hazard analysis, etc), it any, needed to be changed, and why?
  \item What are the limitations of your solution?  Put another way, given
  unlimited resources, what could you do to make the project better? (LO\_ProbSolutions)
  \item Give a brief overview of other design solutions you considered.  What
  are the benefits and tradeoffs of those other designs compared with the chosen
  design?  From all the potential options, why did you select the documented design?
  (LO\_Explores)
\end{enumerate}


\end{document}